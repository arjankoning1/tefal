\begin{samplecase}
{\bf 20 MeV neutron general purpose file for ${}^{72}$Ge}\newline
This sample case concerns a conventional neutron data library, i.e. a
general purpose library up to an energy of 20 MeV for $^{72}$Ge.
A total of 4 files are needed for this run: the TALYS input file
{\em talys.inp}, the energy grid {\em energies}, the optical model file
{\em ge.omp} that is specified in the TALYS input file (for this particular case, see input file below) and the
TEFAL input file {\em tefal.inp}.
First, a TALYS calculation is performed \newline

{\bf talys $<$ talys.inp $>$ talys.out}\newline

where the input file {\em talys.inp} looks as follows:

\VerbatimInput{\samples n-Ge072-20MeV-gpf/org/talys.inp}

Note the essential input line with {\bf endf y}. For the rest, the TALYS input
consists of some adjustable nuclear model parameters and some parameters
determining the precision. We refer to the TALYS manual for the exact
meaning of these keywords.
The TEFAL input file is empty for this sample case, though needs to exist. TEFAL is executed\newline

{\bf tefal $<$ tefal.inp $>$ tefal.out}\newline

After this run, the most important result is a file called {\em n-Ge072.gpf}, which
is the complete ENDF-6 formatted file. The top of this file shows the well-known ENDF format:

\VerbatimInput{\samples n-Ge072-20MeV-gpf/tex/n-Ge072.head}

In addition, there is the
standard TEFAL output file {\em tefal.out} which looks as follows

\VerbatimInput{\samples n-Ge072-20MeV-gpf/org/tefal.out80}

Note that a table with all keywords is given,
not only the ones that you have specified in the input file (in this case: none), but also all the
defaults that are set automatically. The corresponding Fortran variables are
also printed, together with a short explanation of their meaning. This table
can be helpful as a guide to change further input parameters for a next run.
Finally, at the end of the output it is mentioned that a file {\em tefal.clean} is made which contains a list of cleaned up double points.

The ENDF file of this sample case shows the default ENDF-6 procedures we have
used (since {\em tefal.inp} is empty): all open channels are included
(many MT numbers), inelastic scattering angular distributions for discrete
levels are stored in MF4/MT51, etc, MF6 is used for secondary distributions of
partial channels, and MF12 is used for discrete level photon production.
\end{samplecase}
